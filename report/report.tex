\documentclass[a4paper]{article}
\usepackage[brazil]{babel} 
%\usepackage[english]{babel}
\usepackage[utf8]{inputenc}
\usepackage{amsmath}
\usepackage{graphicx}

\usepackage[colorinlistoftodos]{todonotes}



\title{Relatório do Projeto de Métodos Numéricos}

\author{Ermano A. Arruda}

\date{\today}

\begin{document}

\begin{titlepage}

\newcommand{\HRule}{\rule{\linewidth}{0.5mm}} % Defines a new command for the horizontal lines, change thickness here

\center % Center everything on the page
 
%----------------------------------------------------------------------------------------
%	HEADING SECTIONS
%----------------------------------------------------------------------------------------

\textsc{\LARGE Universidade Federal de Pernambuco}\\[1.5cm] % Name of your university/college
\textsc{\Large Centro de Informática - CIn}\\[0.5cm] % Major heading such as course name
\textsc{\large Disciplina de Métodos Numéricos}\\[0.5cm] % Minor heading such as course title

%----------------------------------------------------------------------------------------
%	TITLE SECTION
%----------------------------------------------------------------------------------------

\HRule \\[0.4cm]
{ \huge \bfseries Relatório do Projeto de Métodos Numéricos}\\[0.4cm] % Title of your document
\HRule \\[1.5cm]
 
%----------------------------------------------------------------------------------------
%	AUTHOR SECTION
%----------------------------------------------------------------------------------------

\begin{minipage}{0.4\textwidth}
\begin{flushleft} \large
\emph{Autor:}\\
Ermano A. \textsc{Arruda} % Your name
\end{flushleft}
\end{minipage}
~
\begin{minipage}{0.4\textwidth}
\begin{flushright} \large
\emph{Professor:} \\
Dr. Ricardo \textsc{Martins} % Supervisor's Name
\end{flushright}
\end{minipage}\\[4cm]

% If you don't want a supervisor, uncomment the two lines below and remove the section above
%\Large \emph{Author:}\\
%John \textsc{Smith}\\[3cm] % Your name

%----------------------------------------------------------------------------------------
%	DATE SECTION
%----------------------------------------------------------------------------------------

{\large \today}\\[3cm] % Date, change the \today to a set date if you want to be precise

%----------------------------------------------------------------------------------------
%	LOGO SECTION
%----------------------------------------------------------------------------------------

%\includegraphics{Logo}\\[1cm] % Include a department/university logo - this will require the graphicx package
 
%----------------------------------------------------------------------------------------

\vfill % Fill the rest of the page with whitespace

\end{titlepage}

\tableofcontents

\maketitle



\begin{abstract}
Your abstract.
\end{abstract}

\section{Introdução}

Your introduction goes here! Some examples of commonly used commands and features are listed below, to help you get started. If you have a question, please use the help menu (``?'') on the top bar to search for help or ask us a question.

\section{Métodos de Passo Único}
\label{sec:examples}



\subsection{Euler}

Metodo bla bla bla


\begin{figure}[h]
\centering
\includegraphics[width=0.3\textwidth]{watched.jpg}
\caption{\label{fig:frog}This frog was uploaded to writeLaTeX via the project menu.}
\end{figure}

\begin{table}[h]
\centering
\begin{tabular}{l|r}
Item & Quantity \\\hline
Widgets & 42 \\
Gadgets & 13
\end{tabular}
\caption{\label{tab:widgets}An example table.}
\end{table}


\subsection{Euler Inverso}

Metodo euler inverso bla bla bal

\subsection{Euler Aprimorado}

Metodo Euler aprimorado bla balb alba

\subsection{Runge-Kutta}

Metodo runge kutta bla blab alba

\subsection{Série de Taylor de Três Termos}

bla bla bla serie de taylor de tres termos blab albalbla

\section{Métodos de Passo Múltiplo}

Aqui fica uma explicacao sobre metodos de passo multiplo blab albalba

\subsection{Adams-Bashforth}


Familia Adams albalbalblalba


\subsection{Adams-Multon}

Adams multon ohohohohhooh!

\subsection{Predição e Correção}

Previsao e correcao lol

\subsection{Diferenciação Inversa}

Derivando o DI lol hohohoo Metodos de diferenciacao inversa


Comments can be added to the margins of the document using the \todo{Here's a comment in the margin!} todo command, as shown in the example on the right. You can also add inline comments:

\todo[inline, color=green!40]{This is an inline comment.}

\subsection{How to Include Figures}

First you have to upload the image file (JPEG, PNG or PDF) from your computer to writeLaTeX using the upload link the project menu. Then use the includegraphics command to include it in your document. Use the figure environment and the caption command to add a number and a caption to your figure. See the code for Figure \ref{fig:frog} in this section for an example.

\begin{figure}
\centering
\includegraphics[width=0.3\textwidth]{watched.jpg}
\caption{\label{fig:frog}This frog was uploaded to writeLaTeX via the project menu.}
\end{figure}

\subsection{How to Make Tables}

Use the table and tabular commands for basic tables --- see Table~\ref{tab:widgets}, for example.

\begin{table}
\centering
\begin{tabular}{l|r}
Item & Quantity \\\hline
Widgets & 42 \\
Gadgets & 13
\end{tabular}
\caption{\label{tab:widgets}An example table.}
\end{table}

\subsection{How to Write Mathematics}

\LaTeX{} is great at typesetting mathematics. Let $X_1, X_2, \ldots, X_n$ be a sequence of independent and identically distributed random variables with $\text{E}[X_i] = \mu$ and $\text{Var}[X_i] = \sigma^2 < \infty$, and let
\begin{equation}
S_n = \frac{X_1 + X_2 + \cdots + X_n}{n}
      = \frac{1}{n}\sum_{i}^{n} X_i
\end{equation}
denote their mean. Then as $n$ approaches infinity, the random variables $\sqrt{n}(S_n - \mu)$ converge in distribution to a normal $\mathcal{N}(0, \sigma^2)$.

\subsection{How to Make Sections and Subsections}

Use section and subsection commands to organize your document. \LaTeX{} handles all the formatting and numbering automatically. Use ref and label commands for cross-references.

\subsection{How to Make Lists}

You can make lists with automatic numbering \dots

\begin{enumerate}
\item Like this,
\item and like this.
\end{enumerate}
\dots or bullet points \dots
\begin{itemize}
\item Like this,
\item and like this.
\end{itemize}
\dots or with words and descriptions \dots
\begin{description}
\item[Word] Definition
\item[Concept] Explanation
\item[Idea] Text
\end{description}

We hope you find write\LaTeX\ useful, and please let us know if you have any feedback using the help menu above.

\end{document}